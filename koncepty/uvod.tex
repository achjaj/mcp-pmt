\section{Introduction}

\subsection{PMTs}
A photomultiplier tube is an device capable of converting weak EM signal to strong electric current.
Usability of such a device is quite broad, some PMTs can detect signal from visible light while other
are used to detect soft X-rays.

Common PMTs are made out of glass vacuum tubes which contains three main parts: photocatode, dynodes
and anode. On photocatode the incident foton is converted to electron via photoeffect, so it is
made out of a material with suitable work function depending on the radiation one wants to detect.
PMTs contains multiple dynodes which are just electrodes. High voltage is put between individual
dynodes and also between the photocatode and first dynode.

\subsection{PMTs operation}
When electron is ejected from the photocatode, it is accelerated by the high voltage and when it
hits the first dynode, an secondary electrons are emitted from it. These electrons are also accelerated
and they hit the next dynode, from which more electrons are emitted and so on. The electrons from
the last dynode are collected by the anode and strong electric signal is produced.

\subsection{MCP}
Beside the conventional PMTs there are also so called micro channel plates. These are 2D arrays
of large number of glass tubes, also called channels, which are parallel to each other. The channels
can be either normal to the surface of the plate or slightly slanted. The typical inner diameter of
the channels is around $\SIrange{6}{100}{\micro\metre}$. The channels are made from non conducting
material, so the inner walls needs to be treated in such a way that they became semiconducting and
have appropriate secondary emisive properties. By coating the input and output surface of MCP, the
input and output electrodes are created and a parallel electrical connection between the channels
is ensured. The resistance between the channels is in order of $\SI{e9}{\ohm}$ and each channel
is working as independent electron multiplier.

\subsection{MCP operation}
The treated inner walls behaves as a continuous dynode, so when an electron, or any other capable
particle (including photons), hit the wall, the secondary electrons are emitted and the avalanche
propagates through the channel until a large number of electrons fly out. Such MCP is directly
sensitive to charged particles or ultraviolet and soft X-ray photons. By placing an appropriate
photocatode in front of the input surface, one can detect also an photons from visible portion
of the light spectrum. The multiplication factor of an MCP is of $\numrange{e4}{e7}$ however they
main advantages are great time resolution, which is less than $\SI{700}{\pico\second}$, high
spatial resolution in two dimensions and are stable in a magnetic field.

\subsection{Ionic feedback}
When electrons are travelling through channel, they are accelerated by axial electric field and
upon hitting the walls, they produce more electrons. However these accelerated electrons can also
hit molecules of residual gas and produce positively charged ions, which are then accelerated back
to the input surface. Here they can either hit the walls or the photocatode and create more electrons.
This is called ionic feedback and it can significantly increase the gain of MCP. On the other hand,
the ionic feedback strongly depends on the gas pressure, which may negatively influence some
measurements.

\subsection{Gain}
(Gain of MCP is a function of multiple parameters: applied voltage,
geometry, used materials, gas pressure and puls repetition rate.)

When the first electron hits the wall, it produces $\delta$
additional electrons which then produces $\delta^2$ electrons and so on. The final number of $e^-$
which leaves the channel is called gain and it is equal to $G = \delta^n$, where $n$ is the total
number of stages or collisions. We can expect that the number $\delta$ depends on the collision
energy $T$ and we can write this dependence as $\delta = KT$, where $K$ is appropriate constant.
So, when we increase the applied voltage, the collision energy $T$ also increases and we can expect
increase in gain. However, with the increased voltage the number of stages $n$ decreases,
which means that the final dependence of gain on the applied voltage should excerpt some maximum,
after which the gain starts to decrease. The final relation for gain is
\begin{equation}
    G = \left(\frac{KU^2}{4T_0\alpha^2}\right)^{4T_0\alpha^2/U},
\end{equation}
where $U$ is applied voltage, $T_0$ is initial energy of secondary electron and $\alpha$ is length
to diameter ratio. This is a very simplified model which does not account for ionic feedback and it
assumes that the $e^-$ hit the walls perpendiculary. In reality we see no decrease of gain, but
it saturates at much higher values as is the theoretical maximum.
